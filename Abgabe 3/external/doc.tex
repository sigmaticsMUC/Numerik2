
\documentclass[12pt,titlepage]{article}
\usepackage[utf8x]{inputenc}
\usepackage[ngerman]{babel}
\usepackage{cite}
\usepackage[a4paper,lmargin={2.5cm},rmargin={2.5cm},
tmargin={2.5cm},bmargin = {4cm}]{geometry}
\usepackage{amsmath}
\usepackage{enumitem}
\usepackage{graphicx}
\usepackage{listings}
\usepackage{color}
\usepackage{caption,tabularx,booktabs}
\usepackage{multirow}
\usepackage{here}
\usepackage{float}
\usepackage{fancyhdr}
\usepackage{lmodern}
\usepackage{blindtext}

\begin{titlepage}
\author{Klas Benjamin, Knoll Alexander }
\title{Iterative Verfahren}
\date{\today}
\end{titlepage}

\pagestyle{fancy}
%
\makeatletter
\let\runauthor\@author
\lhead{Numerik 2}
\chead{}
\rhead{\includegraphics[height=50px]{hmlogo}}
\setlength\headheight{40px}
\setlength{\parindent}{0pt} % Neue Absätze werden nicht eingerückt

%
\lfoot{}
\cfoot{\thepage}
\rfoot{\runauthor}
\makeatother
%%%%%%%%%%%%%%%%%%%%%%%%%%%%%%%%%%%%%%%%%%%%%%%%%
%     Neue Befehle werden hier definiert        %
%%%%%%%%%%%%%%%%%%%%%%%%%%%%%%%%%%%%%%%%%%%%%%%%%

\renewcommand{\headrulewidth}{0.4pt}
\renewcommand{\footrulewidth}{0.4pt}

\lstloadlanguages{Python}
\lstset{numbers=left,stepnumber=1,frame=single,language=Python,
        basicstyle=\scriptsize\ttfamily,numberstyle=\scriptsize,
        commentstyle=\upshape\ttfamily,
        numbersep=7pt,tabsize=2,breaklines=false,
        morecomment=[l]{//},showtabs=false,showspaces=false,
        showstringspaces=false,extendedchars=true,inputencoding={utf8},
        keywordstyle=\bfseries\color{darkblue},stringstyle=\color{darkred}}
\newcommand{\pp}[1]{\phantom{#1}}

% diese zeile ermöglicht aufzählungen über buchstaben
% \renewcommand{\theenumi}{\Alph{enumi}}

\newcommand\Section[1]{ %
  \addtocontents{toc}{\protect\setcounter{tocdepth}{0}}
  \subsubsection*{#1}
  \addtocontents{toc}{\protect\setcounter{tocdepth}{3}}}
\definecolor{darkblue}{rgb}{0,0,.6}
\definecolor{darkred}{rgb}{.6,0,0}
\definecolor{darkgreen}{rgb}{0,.6,0}
\definecolor{red}{rgb}{.98,0,0}
\renewcommand{\refname}{Literaturverzeichnis}
\renewcommand{\contentsname}{Inhaltsverzeichnis}

\begin{document}

\maketitle

\part*{Euler-Bernoulli-Balken}

	Der Euler-Bernoulli-Balken ist ein einfaches Modell für eine Biegevorgang auf Grund von
	Spannung. Bezeichnet $y(x)$ für $0 ≤ x ≤ L$ die vertikale Auslenkung, so gilt
	$EIy (x) = f (x)$ wobei $E$ eine Material-Konstante und $I$ das Trägheitsmoment ist. 
	$f(x)$ beschreibt als Kraft pro Einheitslänge die Beladung des Balkens. Durch Diskretisierung erhält man aus der Differentialgleichung ein lineares Gleichungsystem, das hier iterativ gelöst werden soll.
	
	Betrachtet wird dabei ein Stahlträger der Länge $L = 10m$ mit Tiefe $d = 5cm$ und Breite $b = 10cm$. Die Dichte von Stahl ist ungefähr $7850 \frac{kg}{m^3}$, 
	$E = 2 \cdot 10^11 \frac{N}{m^2}$, $I = \frac{bd^3}{12}$

\section{Unbelasteter Balken}

	Zunächst soll ein an beiden Seiten aufliegender Balken untersucht werden.
	Somit gilt $y(0) = y'(x0) = y(L) = y'(L) = 0 $.\newline
	Um entscheiden zu können welches Verfahren zur Lösung des Problems geeignet ist, müssen diese auf Konvergenz untersucht werden. Das Problem ist gegeben durch
	\begin{equation}
		Ax = \frac{h^4}{EI}f,
	\end{equation}
	
	wobei $h = \frac{L}{n+1}$ und $f = g \cdot f(x)$ gilt. Um zu prüfen ob eines der Verfahren konvergiert muss $A$ auf gewisse eigenschaften überprüft werden. Die Verfahren Konvergieren wenn eine der folgenden bedingungen erfüllt sind.
	
	\begin{itemize}
		\centering
		\item Jacobi Verfahren
		\begin{itemize}
			\centering	
			\item Diagonaldominanz
		\end{itemize}
		\item Gauß-Seidel Verfahren
		\begin{itemize}
			\centering
			\item Diagonaldominanz
			\item Positiv-definit
		\end{itemize}
	\end{itemize}
	
	\subsection{Konvergenzuntersuchung}
		
		\subsubsection{Diagonaldominanz}
			Beide Verfahren konvergieren sobald $A$ diagonaldominant ist. Speziel
			für die gegebene Matrix lässt sich sagen, dass dies nicht der Fall ist.
			Für Strikte Diagonaldominanz muss gelten
			\begin{equation*}
				\sum_{j = 1, j \neq i}^{n} |a_{ij}| < |a_{ii}|,
			\end{equation*}
		
			bzw. für Schwache Diagonaldominanz
			\begin{equation*}
				\sum_{j = 1, j \neq i}^{n} |a_{ij}| \leq |a_{ii}|.
			\end{equation*}
		
			Somit wird ersichtlich dass aufgrund der beiden ersten und leitzen Zeilen
			$A$ nicht Diagonaldominant ist und somit dass Jacobi Verfahren nicht
			konvergiert.

		\subsubsection{Positiv Definit}
		
			Eine Matrix M ist positiv Definit genau dann wenn alle Eigenwerte größer als $0$ sind. Zur bestimmung der Eigenwerte wird folgende Gleichung gelöst
			\begin{equation*}
				det(A - E\lambda) = 0.
			\end{equation*}
			
			Glücklicherweise kürzen sich viele Terme da auf den meisten Diagonalen Null Einträge enthalten sind. Somit erzeugt einzig die Hauptdiagonale einen Term, für beliebig große Matrizen dieser Form,
			der dargestellt werden kann als
			\begin{equation*}
				(12-\lambda)\cdot\prod_{i=1}^{n+1}(6-\lambda) \cdot (12-\lambda) = 0
			\end{equation*}
			
			Somit sind nur die Eigenwerte $\lambda = 12$ und $\lambda = 6$ lösungen. Da diese Eigenwerte positiv sind folgt, dass das Gaus-Seidel Verfahren für beliebig große Matrizen $A$ konvergiert.
\Section{Belasteter Balken}


\Section{Herleitung der Diskretisierung}

\Section{Eingespannter Balken}

\Section{Vergleich der Tragfähigkeit}

\Section{SOR-Verfahren}

\Section{cg-Verfahren}

\newpage
\bibliography{literatur}
\bibliographystyle{agsm}

\begin{thebibliography}{xxxxxxxxxxxxxxxxxxxxxxxxxxxxxxxxxxxxxx}

\end{thebibliography}

\end{document}